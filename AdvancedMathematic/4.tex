\chapter{微分中值定理与Taylor公式}

\section{微分中值定理}

\begin{theorem}[罗尔中值定理]
    简单说就是只要两点函数值相等, 就有一点的切线平行与x轴.
    \begin{itemize}
        前提条件: 
        \item $f(x)$在$[a, b]$上连续
        \item $f(x)$在$(a, b)$上可导
        关键条件: 
        \item 如果有$f(a) = f(b)$
        结论:
        \item 那么在$(a, b)$上至少存在一个$\xi$, 使得$f'(\xi) = 0$
    \end{itemize}
\end{theorem}

\begin{proof}
    假设$f(x)$在$[a, b]$上连续且在$(a, b)$上可导,且有$f(a) = f(b)$。
    \begin{itemize}
        \item 如果$f(x)$是常数函数,定理显然成立。
        \item 如果$f(x)$不是常数函数,那么在$[a, b]$上必定存在至少一个最大值或最小值。不妨设$x = c$是$f(x)$在$[a, b]$上的最大值。
        \item 左极限lim$_{x \to c^-} f(x) / (x-c) $>=0, 右极限lim$_{x \to c^+} f(x) / (x-c) $<=0。
        \item 所以$f'(c) = 0$。
    \end{itemize}
\end{proof}

\begin{theorem}[拉格朗日中值定理]
    就是对罗尔中值定理进行旋转. 
    简单说就是函数在某个区间上的平均速度等于某个点的瞬时速度.
    \begin{itemize}
        \item $f(x)$在$[a, b]$上连续
        \item $f(x)$在$(a, b)$上可导
        \item 那么在$(a, b)$上至少存在一个$\xi$, 使得$f'(\xi) = \frac{f(b) - f(a)}{b - a}$
    \end{itemize}
\end{theorem}

\begin{proof}
    对函数进行旋转, 使得$f(a) = f(b) = 0$, 然后使用罗尔中值定理.
\end{proof}

\begin{proof}
    如果不旋转坐标轴也可以. 关键在于如何利用罗尔中值定理证明. 也就是如何寻找$g'(x) = 0$且$g(a) = g(b)$. 
    观察拉格朗日定理的结论, 可以看出$f(b) - f(a) = f'(c)(b - a)$
    进行移项可得$( f(b) - f(a))/(b -a) - f'(c) = 0 = g'(c)$
    对上式进行积分可得$( f(b) - f(a))/(b -a)*x - f(x) = g(x)$
    观察可知可以尝试把分式中的分母消去. 修改为$( f(b) - f(a))/(b -a)*(x - a)- f(x) = g(x)$
    这个变换的妙处在于x分别取a,b时, 一个消去分母, 一个消去整个分式, 保证了算式的工整. 
    检查发现符合罗尔中值定理. 
\end{proof}

\begin{corollary}
    连续且可导的函数的导函数在某个区间恒为0, 那么这个函数就是一个常数函数.
\end{corollary}

\begin{proof}
    利用拉格朗日中值定理, 选择任意两个点, $f'(c) = 0 = (f(x) - f(a))/(x - a) $
    显然可得$f(x) = f(a)$
\end{proof}

\begin{corollary}
    定义了函数的单调性, 
\end{corollary}